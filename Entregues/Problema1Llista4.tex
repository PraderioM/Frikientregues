\documentclass[a4paper,10pt]{article}
\usepackage[utf8]{inputenc}
\usepackage{amsmath}
\usepackage{amssymb}
\usepackage[left=1.5 cm,top=2.5cm,right=1.5cm,bottom=2.5cm]{geometry}
\renewcommand{\*}{\cdot}
\renewcommand{\epsilon}{\varepsilon}
\title{Problema 1 llista 4}
\author{Marco Praderio 1361525}
\date{}
\begin{document}
\maketitle
\paragraph*{Sigui $P_4(x)$ el polinomi interpolador d'una funció $f$ en els nodes $a-2h$, $a-h$, $a$, $a+h$, $a+2h$. Derivant $P_4(x)$ obtenim l'anomenada
\textit{fórmula del cinc punts} per a aproximar $f'(x)$:}
$$f'(a)\approx \frac{1}{12h}(f(a-2h)-8f(a-h)+8f(a+h)-f(a+2h))$$
\paragraph*{Doneu una fórmula per a l'error $f'(a)-P'_4(a)$.\\\phantom{.}\\}

Desenvolupant per Taylor al voltant del punt $a$ la funció $f$\footnote{estem suposant que la funció $f$ és prou regular com perquè tingui sentit lo que escrivim} obtenim:
\begin{equation}
\label{Taylor}
\begin{split}
f(x)&=\sum_{n=0}^{\infty}\frac{f^{n)}(a)}{n!}(x-a)^n\\
&=f(a)+f^{I)}(a)(x-a)+\frac{f^{II)}(a)}{2}(x-a)^2+\frac{f^{III)}(a)}{6}(x-a)^3+\frac{f^{IV)}(a)}{24}(x-a)^4+\frac{f^{V)}(a)}{120}(x-a)^5+O(x-a)^6 
\end{split}
\end{equation}
Si ara substituïm aquesta formula en la aproximació de $f'(a)$ esmentada obtenim:
\begin{equation*}
 \begin{split}
 P'_4(a)&=\frac{1}{12h}(f(a-2h)-8f(a-h)+8f(a+h)-f(a+2h))=\\
 &=\frac{1}{12h}\Bigg(f(a)-2hf^{I)}(a)+2h^2f^{II)}(a)-\frac{4}{3}h^3f^{III)}(a)+\frac{2}{3}h^4f^{IV)}(a)-\frac{4}{15}h^5f^{V)}(a)+O_{-2h}(-2h)^6-\\
 &\phantom{=}-8\left(f(a)-hf^{I)}(a)+\frac{h^2}{2}f^{II)}(a)-\frac{h^3}{6}f^{III)}(a)+\frac{h^4}{24}f^{IV)}(a)-\frac{h^5}{120}f^{V)}(a)+O_{-h}(-h)^6\right)+\\
 &\phantom{=}+8\left(f(a)+hf^{I)}(a)+\frac{h^2}{2}f^{II)}(a)+\frac{h^3}{6}f^{III)}(a)+\frac{h^4}{24}f^{IV)}(a)+\frac{h^5}{120}f^{V)}(a)+O_{h}(h)^6\right)-\\
 &\phantom{=}-f(a)-2hf^{I)}(a)-2h^2f^{II)}(a)-\frac{4}{3}h^3f^{III)}(a)-\frac{2}{3}h^4f^{IV)}(a)-\frac{4}{15}h^5f^{V)}(a)+O_{2h}(2h)^6\Bigg)=\\
 &=\frac{1}{12h}\left(+12hf^{I)}(a)-\frac{2}{5}h^5f^{V)}(a)+O'(h)^6\right)=f'(a)-\frac{1}{30}h^4f^{V)}(a)+O(h)^5
 \end{split}
\end{equation*}
Es compleix per tant que
$$f'(a)-P'_4(a)=\frac{1}{30}h^4f^{V)}(a)+O(h)^5$$
\paragraph*{Si denotem}
$$F(h)=\frac{1}{12h}(f(a-2h)-8f(a-h)+8f(a+h)-f(a+2h))$$
\paragraph*{demostreu que el seu desenvolupament asimptòtic és}
$$F(h)=f'(a)+b_1h^4+b_2h^6+b_3h^8+\cdots$$
\phantom{.}\newline
Reescrivint $F(h)$ i aplicant el desenvolupament en sèrie de potències que es mostra en (\ref{Taylor}) obtenim
\begin{equation*}
 \begin{split}
  F(h)&=\frac{1}{12h}(f(a-2h)-8f(a-h)+8f(a+h)-f(a+2h))=\frac{(f(a-2h)-f(a+2h))-8(f(a-h)-f(a+h))}{12h}=\\
  &=\frac{\left(\sum\limits_{n=0}^{\infty}\frac{f^{n)}(a)}{n!}(-2h)^n-\sum\limits_{n=0}^{\infty}\frac{f^{n)}(a)}{n!}(2h)^n\right)-8\left(\sum\limits_{n=0}^{\infty}\frac{f^{n)}(a)}{n!}(-h)^n-\sum\limits_{n=0}^{\infty}\frac{f^{n)}(a)}{n!}(h)^n\right)}{12h}=\\
  &=\frac{\sum\limits_{n=0}^{\infty}\left(\frac{f^{n)}(a)}{n!}\left((-2h)^n-(2h)^n\right)\right)-8\sum\limits_{n=0}^{\infty}\left(\frac{f^{n)}(a)}{n!}\left((-h)^n-(h)^n\right)\right)}{12h}=\\
  &=\frac{-\sum\limits_{n=0}^{\infty}\frac{f^{2n+1)}(a)}{(2n+1)!}(2h)^{2n+1}+8\sum\limits_{n=0}^{\infty}\frac{f^{2n+1)}(a)}{(2n+1)!}(h)^{2n+1}}{6h}=
  \frac{1}{6}\sum\limits_{n=0}^{\infty}\frac{f^{2n+1)}(a)}{(2n+1)!}(8-2^{2n+1})h^{2n}=\\
  &=f'(a)+0+\frac{1}{6}\sum\limits_{n=2}^{\infty}\frac{f^{2n+1)}(a)}{(2n+1)!}(8-2^{2n+1})h^{2n}=f'(a)+b_1h^4+b_2h^6+b_3h^8+\dots
 \end{split}
\end{equation*}
On hem definit $b_n=\frac{f^{2n+3)}(a)}{6(2n+3)!}(8-2^{2n+3})$
\paragraph*{Apliqueu l'extrapolació de Richardson a $F(h)$ per calcular $f'(2)$ amb $f(x)=x\*sin(x)$ (preneu $h_0=0,1$ i $q=1/2$).\\\phantom{.}\\}
Abans de aplicar l'extrapolació de Richardson en el 0 calculem explícitament $f'(2)$ d'aquesta manera obtenim el següent resultat:
$$f'(x)=\sin(x)+x\cos(x)\Rightarrow f'(2)=\sin(2)+2\cos(2)\approx0.0770037537314$$
Usant aquest valor com a valor real de $f'(2)$ mirem (executant el programa 'Problema1Llista4') a partir de quin valor de $h_i=q^i\*h_0$ la avaluació de $F$ en el 
punt $h_i$ comença a donar errors degut a operacions amb punt flotant. Obtenim d'aquesta manera que la funció $F$ dóna la millor aproximació de $f'(2)$ en el vuitè
valor o sigui avaluant-la en el punt $h_7=10^{-1}2^{-7}$. Aplicant Extrapolació de Richardson (sempre fent servir el programa 'Problema1Llista4') amb les 8 dades inicials
donades per $F(h_i)$ amb $i=0,\dots,7$ obtenim l'aproximació de $f'(2)$ donada per
$$f'(2)\approx0.0770037537313$$
Que te 12 decimals iguals a la anterior aproximació de la derivada obtinguda.
\end{document}