\documentclass[a4paper,10pt]{article}
\usepackage[utf8]{inputenc}
\usepackage{amsmath}
\usepackage{amssymb}
\usepackage[left=1.5 cm,top=2.5cm,right=1.5cm,bottom=2.5cm]{geometry}
\renewcommand{\*}{\cdot}
\newcommand{\R}{\mathbb{R}}
\newcommand{\N}{\mathbb{N}}
\renewcommand{\d}{\text{d}}
\renewcommand{\a}{\alpha}
\newcommand{\e}{\varepsilon}
\renewcommand{\b}{\beta}
\newcommand{\x}{\times}

\title{Funció distancia}
\author{Marco Praderio 1361525}
\date{}
\begin{document}
\maketitle
sigui $E$ un espai mètric amb la funció distància d i sigui $E\x E$ l'espai mètric amb la distància producte $\d\x\d$ volem demostrar que la funció distància 
$\d:E\x E\to\R^+$ és una funció continua.\\
Dir que d és continua és equivalent a dir que, per a tot punt $(x_0,y_0)\in E\x E$ es compleix que
\begin{equation*}
\d(x,y)\to\d(x_0,y_0) \text{ quan } (x,y)\to(x_0,y_0)
\end{equation*}
Com que d és una funció distància tenim que
\begin{equation*}
\d(x,y)\le \d(x,x_0)+\d(x_0,y)=\d(x,x_0)+\d(x_0,y_0)+\d(y_0,y)        \Leftrightarrow      \d(x,y)\le\d(x_0,y_0)+(\d(y_0,y)+\d(x,x_0))
\end{equation*}
i, anàlogament
\begin{equation*}
\d(x_0,y_0)\le\d(x_0,x)+\d(x,y)=\d(x_0,x)+\d(x,y)+\d(y,y_0)         \Leftrightarrow      \d(x,y)\ge\d(x_0,y_0)-(\d(x_0,x)+\d(y,y_0))
\end{equation*}
Per tant, com que quan $(x,y)\to(x_0,y_0)$ es compleix que  $x\to x_0$ i que  $y\to y_0$, aleshores, per definició de límit, es compleix
\begin{equation*}
\d(x,x_0)\to0 \text{ quan } (x,y)\to(x_0,y_0)
\end{equation*}
i
\begin{equation*}
d(y,y0)\to0 \text{ quan } (x,y)\to(x_0,y_0)
\end{equation*}
i com que acabem de veure que
\begin{equation*}
\d(x_0,y_0)-(\d(x_0,x)+\d(y,y_0))\le\d(x,y)\le\d(x_0,y_0)+(\d(y_0,y)+\d(x,x_0))
\end{equation*}
aleshores podem assegurar per la regla del Sandwich que
\begin{equation*}
\d(x,y)\to\d(x_0,y_0) \text{ quan } (x,y)\to(x_0,y_0)
\end{equation*}
i, per tant, la funció d és continua.
\end{document}