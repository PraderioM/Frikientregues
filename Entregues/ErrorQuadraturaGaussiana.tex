\documentclass[a4paper,10pt]{article}
\usepackage[utf8]{inputenc}
\usepackage{amsmath}
\usepackage{amssymb}
\usepackage[left=1.5 cm,top=2.5cm,right=1.5cm,bottom=2.5cm]{geometry}
\renewcommand{\*}{\cdot}
\renewcommand{\phi}{\rho}
\newcommand{\R}{\mathbb{R}}
%opening
\title{Formula del error per la quadratura Gaussiana.}
\author{Marco Praderio 1361525}
\date{}
\begin{document}
\maketitle
El nostre objectiu és demostrar que:\\
Siguin $x_1$,$x_2$,$\dots$,$x_n$ els zeros (simples) del polinomi ortogonal $\phi_n$ (de grau $n$) respecte del pes $\omega(x)$ a l'interval $[a,b]$ i sigui $f:[a,b]\to\R$ de
classe $C^{2n}$. Llavors
\begin{equation*}
 \int_a^b\omega(x)f(x)dx-\sum_{i=1}^n\omega_if(x_i)=\frac{f^{(2n)(\xi)}}{(2n)!}<\phi_n,\phi_n>
\end{equation*}
amb $a<\xi<b$.
Per fer-ho aplicarem que la formula de quadratura Gaussiana amb $n$ nodes és exacte per a qualsevol polinomi de grau mes petit o igual que $2n-1$ i que l'error de
interpolació de polinomis de Hermite obtinguts amb $n+1$ nodes és $f(x)-H_n(x)=\frac{f^{(n+1)}(\xi_x)}{(n+1)!}(x-x_0)^{n_0}\dots(x-x_m)^{n_m}$ on $\xi_x\in<x_0,\dots,x_m,x>$
$x_i$ són les abscisses dels punts fets servir per interpolar i $n_i$ és el nombre de derivades +1 que coneixem d'aquests punts.
\paragraph*{}
Agafem $H_{2n-1}$ el polinomi interpolador de Hermite que interpola $f$ en les $n$ arrels del polinomi ortogonal $\phi_n$ i en la derivada de $f$ en els mateixos punts.
Com que $H_{2n-1}$ interpola $f$ en les $n$ arrels de $\phi_n$ (que denotarem per $x_i$) aleshores es compleix que
$$\sum_{i=1}^n\omega_if(x_i)=\sum_{i=1}^n\omega_iH_{2n-1}(x_i)$$
Si a més a més tenim en compte que
\begin{equation*}
 \begin{split}
  f(x)-H_{2n-1}(x)&=\frac{f^{(2n)}(\xi_x)}{(2n)!}(x-x_0)^2\dots(x-x_m)^2 \Rightarrow \\
  \Rightarrow f(x)&=H_{2n-1}(x)+\frac{f^{(2n)}(\xi_x)}{(2n)!}(x-x_0)^2\dots(x-x_m)^2=
H_{2n-1}(x)+\frac{f^{(2n)}(\xi_x)}{(2n)!}\phi_n^2(x)
 \end{split}
\end{equation*}

Aleshores podem escriure
\begin{equation}
\label{ref}
\begin{split}
\int_a^b\omega(x)f(x)dx-\sum_{i=1}^n\omega_if(x_i)&=\int_a^b\omega(x)\left(H_{2n-1}(x)+\frac{f^{(2n)}(\xi_x)}{(2n)!}\phi_n^2(x)\right)dx-\sum_{i=1}^n\omega_iH_{2n-1}(x_i)=\\
&=\int_a^b\omega(x)H_{2n-1}(x)dx-\sum_{i=1}^n\omega_iH_{2n-1}(x_i)+\int_a^b\omega(x)\frac{f^{(2n)}(\xi_x)}{(2n)!}\phi_n^2(x)dx
\end{split}
\end{equation}
Com que $H_{2n-1}$ és un polinomi de grau menor o igual a $2n-1$ i la formula de quadratura Gaussiana és exacta per a polinomis de grau menor o igual a $2n-1$ aleshores tindrem
$$\int_a^b\omega(x)H_{2n-1}(x)dx=\sum_{i=1}^n\omega_iH_{2n-1}(x_i)\Rightarrow \int_a^b\omega(x)H_{2n-1}(x)dx-\sum_{i=1}^n\omega_iH_{2n-1}(x_i)=0$$
A més a més, com que $f$ és $C^{2n}$ aleshores $f^{(2n)}$ és continua i aleshores podem aplicar el teorema del valor mig per afirmar que existeix $\xi\in(a,b)$ que compleix
$$\int_a^b\omega(x)\frac{f^{(2n)}(\xi_x)}{(2n)!}\phi_n^2(x)dx=\frac{f^{(2n)}(\xi)}{(2n)!}\int_a^b\omega(x)\phi_n^2(x)dx=\frac{f^{(2n)}(\xi)}{(2n)!}<\rho_n,\rho_n>$$
Podem per tant reescriure (\ref{ref}) com
\begin{equation*}
 \begin{split}
  \int_a^b\omega(x)f(x)dx-\sum_{i=1}^n\omega_if(x_i)&=\int_a^b\omega(x)H_{2n-1}(x)dx-\sum_{i=1}^n\omega_iH_{2n-1}(x_i)+\int_a^b\omega(x)\frac{f^{(2n)}(\xi_x)}{(2n)!}\phi_n^2(x)dx\\
  &=0+\frac{f^{(2n)}(\xi)}{(2n)!}<\rho_n,\rho_n>=\frac{f^{(2n)}(\xi)}{(2n)!}<\rho_n,\rho_n>
 \end{split}
\end{equation*}
Tal i com volíem demostrar.
\end{document}