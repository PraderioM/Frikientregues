\documentclass[a4paper,10pt]{article}
\usepackage[utf8]{inputenc}
\usepackage{amsmath}
\usepackage{amssymb}
\usepackage[left=1.5 cm,top=2.5cm,right=1.5cm,bottom=2.5cm]{geometry}
\renewcommand{\*}{\cdot}
\newcommand{\e}{\varepsilon}
\renewcommand{\a}{\alpha}
\renewcommand{\r}{\mathbb{R}}
\renewcommand{\u}{\mu}
%opening
\title{Successió de Sturm}
\author{Marco Praderio 1361525}
\date{}
\begin{document}
 \maketitle
 Volem demostrar que si $p(x)$ té arrels múltiples, amb la notació de les planes anteriors, la Successió donada per
 \begin{equation*}
 \begin{split}
  &q_0(x)=q(x)=\frac{p(x)}{p_{m-1}(x)}\\
  &q_i(x)=\frac{p_i(x)}{p_{m-1}(x)}\text{ per }i=1,\dots,m-1
 \end{split}
\end{equation*}
és de Sturm.\\
Sabem que la successió $\{p_i\}_{i=0}^{m-1}$ està construïda de manera que compleix totes les propietats de una successió de Sturm excepte la propietat 4 que demana que
$p_{m-1}(x)$ no canvii de signe en el interval on es vol estudiar el polinomi. En el cas de la successió $\{q_i\}_{i=0}^{m-1}$ aquesta proprietat si que es compleix en quant
$q_{m-1}=1$. vegem ara que es compleixen les altres proprietats
\begin{itemize}
\item \textbf{$q_0(x)=q(x)$:} Per definició.
 \item \textbf{Si $\a\in\r$ és tal que $q_0(\a)=0$ aleshores $q_0'(\a)\*q_1(\a)$:} Sabem que $q_0(x)=\frac{p_0(x)}{p_{m-1}(x)}$ i, per tant
 $$q'_0(x)=\frac{p_0'(x)p_{m-1}(x)-p'_{m-1}(x)p_0(x)}{p_{m-1}(x)^2}$$
 com que en el punt $x=\a$ es compleix que $q_0(\a)=0$ aleshores $\frac{p_0(\a)}{p_{m-1}(\a)}=q_0(\a)=0 \Rightarrow p_0(\a)=0$ , per tant
 $$q'_0(\a)=\frac{p_0'(x)p_{m-1}(x)-p'_{m-1}(x)p_0(x)}{p_{m-1}(x)^2}=\frac{p_0'(x)}{p_{m-1}(x)}$$
 i podem conloure que $$q'_0(\a)\*q_1(\a)=\frac{p_0'(\a)\*p_1(\a)}{p_{m-1}(x)^2}\ge 0$$ com que $\a$ no és un 0 de $q_1(x)$ ni de $q_0'(x)$ en quant $q_1(\a)=q_0'(\a)$ i la
 funció successió de $q_i$ està construïda de manera que si $\a$ és un 0 de $q_0$ aleshores no és 0 de $q_1$. Per tant podem concloure que $q'_0(\a)\*q_1(\a)>0$
 \item \textbf{per a tot $i=1,\dots,m-2$ es compleix que, si $\a$ és tal que $q_i(\a)=0$ aleshores $q_{i-1}(\a)q_{i+1}(\a)<0$:} Hem vist que si $q_i(\a)=0$ aleshores
 $p_i(\a)=0$ i per tant tenim
 \begin{equation*}
  q_{i-1}(\a)q_{i+1}(\a)=\frac{p_{i-1}(\a)p_{i+1}(\a)}{p_{m-1}(\a)^2}
 \end{equation*}
 i, com que $p_{i-1}(\a)p_{i+1}(\a)<0$ i $p_{m-1}(\a)^2>0$ aleshores $q_{i-1}(\a)q_{i+1}(\a)>0$
\end{itemize}
\end{document}
